\documentclass{beamer}

\usepackage[mode=buildnew]{standalone}
%\usepackage{tikz}
\usepackage{pgfplots}
\usepackage{pgfplotstable}
\pgfplotsset{compat=1.16}
\usepackage{listings}

%% Ugly workaround to get lstlinebgrd to work.
\makeatletter
\let\old@lstKV@SwitchCases\lstKV@SwitchCases
\def\lstKV@SwitchCases#1#2#3{}
\makeatother
\usepackage{lstlinebgrd}
\makeatletter
\let\lstKV@SwitchCases\old@lstKV@SwitchCases

\lst@Key{numbers}{none}{%
    \def\lst@PlaceNumber{\lst@linebgrd}%
    \lstKV@SwitchCases{#1}%
    {none:\\%
     left:\def\lst@PlaceNumber{\llap{\normalfont
                \lst@numberstyle{\thelstnumber}\kern\lst@numbersep}\lst@linebgrd}\\%
     right:\def\lst@PlaceNumber{\rlap{\normalfont
                \kern\linewidth \kern\lst@numbersep
                \lst@numberstyle{\thelstnumber}}\lst@linebgrd}%
    }{\PackageError{Listings}{Numbers #1 unknown}\@ehc}}
\makeatother
\makeatletter
%%%%%%%%%%%%%%%%%%%%%%%%%%%%%%%%%%%%%%%%%%%%%%%%%%%%%%%%%%%%%%%%%%%%%%%%%%%%%%
%
% \btIfInRange{number}{range list}{TRUE}{FALSE}
%
% Test in int number <number> is element of a (comma separated) list of ranges
% (such as: {1,3-5,7,10-12,14}) and processes <TRUE> or <FALSE> respectively

\newcount\bt@rangea
\newcount\bt@rangeb

\newcommand\btIfInRange[2]{%
    \global\let\bt@inrange\@secondoftwo%
    \edef\bt@rangelist{#2}%
    \foreach \range in \bt@rangelist {%
        \afterassignment\bt@getrangeb%
        \bt@rangea=0\range\relax%
        \pgfmathtruncatemacro\result{ ( #1 >= \bt@rangea) && (#1 <= \bt@rangeb) }%
        \ifnum\result=1\relax%
            \breakforeach%
            \global\let\bt@inrange\@firstoftwo%
        \fi%
    }%
    \bt@inrange%
}
\newcommand\bt@getrangeb{%
    \@ifnextchar\relax%
        {\bt@rangeb=\bt@rangea}%
        {\@getrangeb}%
}
\def\@getrangeb-#1\relax{%
    \ifx\relax#1\relax%
        \bt@rangeb=100000%   \maxdimen is too large for pgfmath
    \else%
        \bt@rangeb=#1\relax%
    \fi%
}

%%%%%%%%%%%%%%%%%%%%%%%%%%%%%%%%%%%%%%%%%%%%%%%%%%%%%%%%%%%%%%%%%%%%%%%%%%%%%%
%
% \btLstHL<overlay spec>{range list}
%
% TODO BUG: \btLstHL commands can not yet be accumulated if more than one overlay spec match.
% 
\newcommand<>{\btLstHL}[1]{%
  \only#2{\btIfInRange{\value{lstnumber}}{#1}{\color{orange!30}\def\lst@linebgrdcmd{\color@block}}{\def\lst@linebgrdcmd####1####2####3{}}}%
}%
\makeatother


\usefonttheme{serif}

\title{TikZ/pgfplots tutorial}
\author{Lucas Unnerfelt}
\date{March 2020}

\begin{document}
\begin{frame}
\titlepage
\end{frame}

\section*{Outline}
\begin{frame}
  \tableofcontents
\end{frame}

\lstset{
  rangeprefix=\%\ ,
  includerangemarker=false
}

\section{Basic plotting}
\newcommand{\codesplit}[2]{
\begin{columns}[T]
\begin{column}{.5\textwidth}
  \lstinputlisting[linerange=start-end,
    language=TeX, frame=single, basicstyle=\tiny, #2]{#1.tex}
\end{column}%
\begin{column}{.48\textwidth}
  \includestandalone[width=\textwidth]{#1}
\end{column}%
\end{columns}
}

\begin{frame}
  Relevant packages
  \begin{enumerate}
  \item<1-> tikz
  \item<2-> pgfplots
  \item<3-> pgfplotstable
  \end{enumerate}
  \only<4->{
  All pacakges have extensive documentation on ctan.org
  Pgfplots and pgfplotstable have reasonably large documentations.
  TikZ has more than $1000$ pages. Lots of code and examples.}
\end{frame}

\begin{frame}
  Including packages, configuration.
  \codesplit{example1}{linebackgroundcolor={\btLstHL<1>{4-5}, \btLstHL<2>{8-19}, \btLstHL<3>{9-18}, \btLstHL<4>{10-17}}}
\end{frame}

\begin{frame}
  Example of data file.
  \lstinputlisting[frame=single, basicstyle=\tiny]{data.txt}
  Saved as data.txt.
  It is possible to reconfigure pgfplots(table) to load a lot of different text formats. But I have found it most simple to reformat in MATLAB/Python instead of in \LaTeX and TikZ.
\end{frame}

\begin{frame}
  Save in correct format using Python/numpy

  \lstinputlisting[frame=single, basicstyle=\tiny]{save.py}
  MATLAB
  \lstinputlisting[frame=single, basicstyle=\tiny]{save_data.m}
  
\end{frame}

\begin{frame}
  Loading data file
  \codesplit{example2}{linebackgroundcolor={\btLstHL<1>{9}, \btLstHL<2>{14}}}
\end{frame}

\begin{frame}
  Axis settings
  \codesplit{example3}{linebackgroundcolor={\btLstHL<1>{13-16}, \btLstHL<2>{18, 21, 24}}}
\end{frame}

\begin{frame}
  Legends
  \codesplit{example4}{linebackgroundcolor={\btLstHL<1>{15-16}}}
\end{frame}

\begin{frame}
  Legends
  \codesplit{example5}{linebackgroundcolor={\btLstHL<1>{17}}}
\end{frame}

\begin{frame}
  Multiple plots with labels.
  \codesplit{example6}{linebackgroundcolor={\btLstHL<1>{}, \btLstHL<2>{14-15, 21-22}}}
\end{frame}

\section{Contour plots}
\begin{frame}
  Contour plots
  \begin{itemize}
  \item Should be simple (and is for non-filled contour plots)
  \item Built in pgfplots functionality for filled contour plots only works in Acrobat Reader
  \item Therefore we need workarounds
  \end{itemize}
\end{frame}

\begin{frame}
  Data we will be working with
  \lstinputlisting[frame=single, basicstyle=\tiny]{save_data_contour.m}
\end{frame}

\begin{frame}
  3D plot.
  \codesplit{example7}{}
\end{frame}

\begin{frame}
  Contour plot.
  \codesplit{example8}{}
\end{frame}

\begin{frame}
  Using MATLAB
  \lstinputlisting[frame=single, basicstyle=\tiny]{export_contour_plot.m}
  \codesplit{example9}{}
\end{frame}

\begin{frame}
  Using matlab, filled
  \lstinputlisting[frame=single, basicstyle=\tiny]{export_contour_plot_filled.m}
  \codesplit{example10}{}
\end{frame}

\begin{frame}
  Colorbar
  \lstinputlisting[linerange=start-end,frame=single, basicstyle=\tiny]{export_contour_plot_filled_cb.m}
  \codesplit{example11}{}
\end{frame}

\section{Misc}
\begin{frame}
  Illustrations (inkscape might be a better choice for this)
  \codesplit{example12}{}
\end{frame}
\begin{frame}
  Speed up compilation
  \lstinputlisting[frame=single, basicstyle=\tiny]{code.tex}
\end{frame}

\begin{frame}
  Documentation for pgfplots, tikz, and even pgfplotstable are really good.
\end{frame}

\end{document}
